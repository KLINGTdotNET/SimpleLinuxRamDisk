\section{Einleitung}
\label{sec:einleitung}

Im Rahmen des Moduls \enquote{Systemprogrammierung} war ein Linux-Gerätetreiber zu entwickeln. Wir entschieden uns dafür ein \href{http://en.wikipedia.org/wiki/RAM_drive}{RAM-Drive}; also einen Blockgerätetreiber, zu entwerfen.

In der folgenden Dokumentation werden wir einige Grundlagen erklären, auf die notwendigen Begrifflichkeiten eingehen und den erstellten Quellcode analysieren. Außerdem wird die Einbindung und die Nutzung des \enquote{RAM-Drive} an einem konkreten Beispiel demonstriert.

Als Entwicklungs- und Testsystem wurde ein 64-Bit \href{http://www.linuxmint.com/}{Linux Mint} mit Kernel-Release \textbf{3.5.0-21-generic}\footnote{Das verwendete Kernel-Release kann mit \texttt{uname -r} abgerufen werden} genutzt.